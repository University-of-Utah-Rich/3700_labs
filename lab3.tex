\documentclass[12pt]{labmanual}


\title{Lab Assignment 3}
\author{}
\date{}

% Set custom header
\setlength{\headheight}{15.2pt}
\pagestyle{fancy}
\fancyhf{}
\fancyhead[R]{ECE/CS 3700}
\fancyhead[L]{Lab 3: Combinational Design Using Always Blocks}

\begin{document}
\maketitle

\section{Introduction}

In this one-week laboratory exercise, you will gain hands-on experience designing combinational circuits using Verilog's always blocks. While previous labs focused on gate-level primitives and continuous assign statements, this lab introduces behavioral modeling using \keyword{always @()} constructs.

\begin{extra}[frametitle={Required Reading}]
Before beginning this lab, review Textbook Section 4.6: Verilog Operators
\end{extra}

The primary objectives of this lab are to:
\begin{enumerate}
    \item Design and implement a configurable 4-bit comparator using always blocks
    \item Practice behavioral modeling techniques in Verilog
    \item Verify design correctness through simulation and synthesis analysis
\end{enumerate}

Throughout this lab, you will learn how to:
\begin{itemize}
    \item Write combinational logic using always blocks
    \item Handle signed and unsigned number comparisons
    \item Create comprehensive testbenches for verification
    \item Analyze synthesis results for combinational correctness
\end{itemize}

\section{Background: Two's Complement Numbers and Comparison}

\subsection{Two's Complement Representation}

Two's complement is a method for representing signed integers that allows both positive and negative numbers to be handled using the same hardware for addition and subtraction. For a 4-bit number:
\begin{itemize}
    \item The most significant bit (MSB) acts as the sign bit (0 for positive, 1 for negative)
    \item Positive numbers are represented in standard binary (0000 to 0111)
    \item Negative numbers are represented by inverting all bits and adding 1
\end{itemize}

\begin{table}[H]
    \centering
    \begin{tabular}{|c|c|c|}
    \hline
    \textbf{Decimal} & \textbf{Binary} & \textbf{Notes} \\
    \hline
    +7 & 0111 & Maximum positive number \\
    +6 & 0110 & \\
    +5 & 0101 & \\
    +4 & 0100 & \\
    +3 & 0011 & \\
    +2 & 0010 & \\
    +1 & 0001 & \\
    0 & 0000 & Zero \\
    -1 & 1111 & \\
    -2 & 1110 & \\
    -3 & 1101 & \\
    -4 & 1100 & \\
    -5 & 1011 & \\
    -6 & 1010 & \\
    -7 & 1001 & \\
    -8 & 1000 & Minimum negative number \\
    \hline
    \end{tabular}
    \caption{4-bit Two's Complement Number Representation}
    \label{tab:twos_complement}
\end{table}

\subsection{Comparison Through Subtraction}

\begin{important}[frametitle={Key Concept}]
The most reliable way to compare two numbers A and B is to perform subtraction (A - B) and examine the result:
\begin{itemize}
    \item If (A - B) > 0: A > B
    \item If (A - B) = 0: A = B
    \item If (A - B) < 0: A < B
\end{itemize}
This approach works for both unsigned and two's complement numbers!
\end{important}

\subsection{Understanding Subtraction Results}

When performing A - B:
\begin{enumerate}
    \item For unsigned numbers:
    \begin{itemize}
        \item The presence of a carry-out indicates A $\geq$ B
        \item The absence of a carry-out indicates A < B
    \end{itemize}
    \item For two's complement numbers:
    \begin{itemize}
        \item The sign bit of the result indicates the relationship
        \item A zero result indicates equality
    \end{itemize}
\end{enumerate}

\begin{extra}[frametitle={Implementation Example}]
Consider comparing A = 0101 (5) and B = 0011 (3):

\textbf{Unsigned Subtraction:}
\begin{verbatim}
  0101 (A)
- 0011 (B)
-------
  0010 with carry-out = 1
\end{verbatim}
Carry-out = 1 indicates A > B

\textbf{Two's Complement Subtraction:}
\begin{verbatim}
  0101 (A)
- 0011 (B)
-------
  0010 (positive result)
\end{verbatim}
Positive result indicates A > B
\end{extra}

\begin{important}[frametitle={Design Consideration}]
Your comparator can use a single subtraction operation and then interpret the result based on:
\begin{itemize}
    \item The control input c
    \item The carry-out (for unsigned mode)
    \item The sign bit of the result (for signed mode)
\end{itemize}
\end{important}

\subsection{Implementation Strategy}

To implement the comparator:
\begin{enumerate}
    \item Perform the subtraction A - B
    \item Based on control input c:
    \begin{itemize}
        \item c = 0: Use carry-out to determine relationship
        \item c = 1: Use result's sign bit to determine relationship
    \end{itemize}
    \item Set F1, F2, F3 based on the interpretation
\end{enumerate}

\begin{question}[Subtraction-Based Comparison]
Work through these examples:
\begin{enumerate}
    \item Calculate A - B for A = 1100, B = 0100:
    \begin{itemize}
        \item Show the subtraction process
        \item Interpret the result for both unsigned and signed modes
        \item Determine F1, F2, F3 for each mode
    \end{itemize}
    \item Repeat for A = 1000, B = 0111
    \item Explain why this method is more reliable than using comparison operators
\end{enumerate}
\end{question}

\begin{extra}[frametitle={Common Pitfalls to Avoid}]
\begin{itemize}
    \item Don't use separate logic paths for signed/unsigned comparisons
    \item Don't rely on Verilog's comparison operators
    \item Remember to handle the equality case (A = B) separately
    \item Ensure proper bit widths in subtraction to capture carry-out
\end{itemize}
\end{extra}

\section{Behavioral Verilog and Always Blocks}

\subsection{Understanding Always Blocks}

An always block is a fundamental construct in Verilog that allows behavioral description of circuits. Unlike continuous assignments (\texttt{assign} statements) which describe concurrent behavior, always blocks describe sequential evaluation of statements.

\begin{important}[frametitle={Always Block Structure}]
An always block contains three crucial parts:
\begin{enumerate}
    \item The sensitivity list: \textbf{\texttt{always @(*)}}
    \item The block opener: \textbf{\texttt{begin}}
    \item The logic to be performed 
    \item The block closer: \textbf{\texttt{end}}
\end{enumerate}
\end{important}

\subsection{Sensitivity Lists}

The sensitivity list tells the compiler which signals should trigger the evaluation of the always block:

\begin{lstlisting}[language=verilog]
// Explicit sensitivity list
always @(a or b or sel) 
// Implicit sensitivity list (preferred)
always @(*)
\end{lstlisting}

\begin{extra}[frametitle={Sensitivity List Best Practices}]
\begin{itemize}
    \item Use \texttt{@(*)} for combinational logic
    \item Include ALL inputs that affect outputs in explicit lists
    \item Missing signals in explicit lists can create latches
    \item For combinational logic, never use clock edges
\end{itemize}
\end{extra}

\subsection{Wires vs Regs}
\textbf{Wires} and \textbf{Regs} are two different ways of passing signals in verilog. They serve different purposes as explained below.
\begin{important}[frametitle={Understanding Nets}]
In Verilog, signals are passed between gates and modules using \textbf{nets}:
\begin{itemize}
    \item \textbf{\texttt{wire}}: Represents physical connections
    \begin{itemize}
        \item Can only be driven by continuous assignments
        \item Cannot store values
        \item Used with \texttt{assign} statements
    \end{itemize}
    \item \textbf{\texttt{reg}}: Represents storage elements
    \begin{itemize}
        \item Required for always block outputs
        \item Does NOT necessarily create flip-flops
        \item Cannot be driven by continuous assignments
    \end{itemize}
\end{itemize}
\end{important}

\begin{extra}[frametitle={Common Misconception}]
The \texttt{reg} keyword does NOT automatically create a register or flip-flop! It simply indicates that the signal can hold a value between always block evaluations. The actual hardware implementation depends on how the signal is used.
\end{extra}

\subsection{Avoiding Common Pitfalls}

\subsubsection{Preventing Latches}

Latches are often created unintentionally when the compiler detects the need to "remember" a value.

\begin{important}[frametitle={Latch Prevention}]
To avoid latches:
\begin{enumerate}
    \item Assign values to ALL outputs in ALL possible conditions
    \item Use complete if-else structures
    \item Include default assignments
    \item Use complete case statements with default
\end{enumerate}
\end{important}

Example of latch creation:
\begin{lstlisting}[language=verilog]
// BAD: Creates a latch
always @(*) begin
    if (sel) 
        y = a;
end

// GOOD: No latch
always @(*) begin
    if (sel)
        y = a;
    else
        y = b;
end
\end{lstlisting}

\subsubsection{Avoiding Multiple Drivers}

A signal in Verilog can only have one driver to prevent hardware conflicts.

\begin{important}[frametitle={Multiple Driver Rules}]
\begin{itemize}
    \item A signal cannot be driven by both an always block and an assign statement
    \item A signal cannot be driven by multiple always blocks
    \item Each reg can only be assigned in one always block
\end{itemize}
\end{important}

Example of multiple driver conflicts:
\begin{lstlisting}[language=verilog]
// BAD: Multiple drivers
assign y = a & b;
always @(*) begin
    y = c & d;  // Conflict!
end

// GOOD: Single driver
always @(*) begin
    y = (sel) ? (a & b) : (c & d);
end
\end{lstlisting}

\subsection{Behavioral Constructs}

Inside always blocks, you can use various behavioral constructs:

\begin{table}[H]
    \centering
    \begin{tabular}{|l|l|}
    \hline
    \textbf{Construct} & \textbf{Usage} \\
    \hline
    if-else & Conditional execution \\
    case & Multiple condition selection \\
    ?: operator & Conditional assignment \\
    blocking assignment (=) & Sequential evaluation \\
    \hline
    \end{tabular}
    \caption{Common Behavioral Constructs}
    \label{tab:constructs}
\end{table}

Example of proper behavioral coding:
\begin{lstlisting}[language=verilog]
always @(*) begin
    // Default assignments prevent latches
    y1 = 0;
    y2 = 0;
    y3 = 0;
    
    case (sel)
        2'b00: y1 = a & b;
        2'b01: y2 = c | d;
        2'b10: y3 = e ^ f;
        default: begin
            y1 = 0;
            y2 = 0;
            y3 = 0;
        end
    endcase
end
\end{lstlisting}

\begin{question}[Behavioral Coding Practice]
Identify and fix the problems in this code:
\begin{lstlisting}[language=verilog]
always @(sel) begin
    if (sel == 1)
        out = in1;
    else if (sel == 0)
        out = in2;
end

always @(*) begin
    out = in3 & in4;
end
\end{lstlisting}
\end{question}

\begin{extra}[frametitle={Design Guidelines}]
\begin{itemize}
    \item Initialize all outputs at the start of the always block
    \item Use explicit width for numbers (4'b0000 vs 0)
    \item Keep logic within a single always block when possible
    \item Use meaningful signal names
    \item Comment your code thoroughly
\end{itemize}
\end{extra}

\section{Design Specifications}

\subsection{Functional Requirements}

Design a configurable 4-bit comparator with the following interface:

\begin{important}[frametitle={Interface Specification}]
\textbf{Inputs:}
\begin{itemize}
    \item \textbf{A[3:0]}: First 4-bit input vector
    \item \textbf{B[3:0]}: Second 4-bit input vector
    \item \textbf{c}: Control input for number interpretation
    \begin{itemize}
        \item c = 0: Treat A,B as unsigned numbers
        \item c = 1: Treat A,B as two's complement numbers
    \end{itemize}
\end{itemize}

\textbf{Outputs:}
\begin{itemize}
    \item \textbf{F1}: Greater than indicator (A > B)
    \item \textbf{F2}: Equal to indicator (A = B)
    \item \textbf{F3}: Less than indicator (A < B)
\end{itemize}
\end{important}

\begin{table}[H]
    \centering
    \begin{tabular}{|c|c|c|c|}
    \hline
    \textbf{Condition} & \textbf{F1} & \textbf{F2} & \textbf{F3} \\
    \hline
    A > B & 1 & 0 & 0 \\
    A = B & 0 & 1 & 0 \\
    A < B & 0 & 0 & 1 \\
    \hline
    \end{tabular}
    \caption{Truth table for comparator outputs}
    \label{tab:truth_table}
\end{table}

\section{Implementation Guidelines}

\begin{extra}[frametitle={Design Approach}]
Your implementation must follow these requirements:
\begin{itemize}
    \item The entire design must be implemented within a single always block
    \item Use appropriate Verilog operators for comparison operations
    \item Consider edge cases for both signed and unsigned comparisons
    \item Ensure mutually exclusive output conditions
\end{itemize}
\end{extra}

Sample code structure:
\begin{lstlisting}[language=Verilog]
module comparator(
    input [3:0] A,
    input [3:0] B,
    input c,
    output reg F1,
    output reg F2,
    output reg F3
);

always @(*) begin
    // Your implementation here
end

endmodule
\end{lstlisting}

\begin{important}[frametitle={Common Pitfalls}]
\begin{itemize}
    \item Ensure all outputs are assigned in all possible conditions to avoid latches
    \item Poor always block implementation may result in incorrect hardware generation
    \item Remember to handle both signed and unsigned cases correctly
\end{itemize}
\end{important}

\section{Assignment Tasks}

\begin{question}[Design Implementation]
\begin{enumerate}
    \item Implement the comparator using a single always block
    \item Document your design decisions and approach
    \item Include commented Verilog code
\end{enumerate}
\end{question}

\begin{question}[Verification]
\begin{enumerate}
    \item Create a comprehensive testbench that verifies:
    \begin{itemize}
        \item All comparison cases (>, =, <)
        \item Both unsigned and signed modes
        \item Corner cases (maximum/minimum values)
        \item Transitions between modes
    \end{itemize}
    \item Provide waveform screenshots showing critical test cases
    \item Document test coverage
\end{enumerate}
\end{question}

\begin{question}[Synthesis Analysis]
\begin{enumerate}
    \item Include schematic capture from synthesis
    \item Report and explain:
    \begin{itemize}
        \item Resource utilization (LABs and LEs/LUTs)
        \item Propagation delay
        \item Verification of combinational implementation (no latches)
    \end{itemize}
\end{enumerate}
\end{question}

\section{Submission Requirements}
Your lab report should include:
\begin{enumerate}
    \item Verilog source code (design and testbench)
    \item Simulation results and waveforms
    \item Synthesis reports and schematics
    \item Written responses to all questions
    \item Analysis of results
\end{enumerate}

\begin{extra}[frametitle={Time Management}]
Estimated time allocation:
\begin{itemize}
    \item Design Implementation: 1 hour
    \item Testbench Development: 1 hour
    \item Simulation and Debug: 1 hour
    \item Synthesis Analysis: 30 minutes
    \item Documentation: 30 minutes
\end{itemize}
Total Estimated Time: 4 hours
\end{extra}

\end{document}