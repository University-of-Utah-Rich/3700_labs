\documentclass[12pt]{labmanual}

\title{Guide to Using the Lab Manual Class}

\begin{document}
\maketitle
\makeheaders
\clearpage
\tableofcontents
\clearpage
\section{Introduction}

This guide demonstrates the key features of the \texttt{labmanual} class for creating consistent laboratory manuals. The class provides specialized environments and commands for structuring your content effectively.

\section{Basic Formatting Commands}

\subsection{Page Setup}
Page setup consists of declaring the document class and text size, adding title and headers, including any extra packages required. The basic template follows.
\begin{codeblock} 
\documentclass[12pt]{labmanual} % Use the labmanual class and 12pt font
\title{Guide to Using the Lab Manual Class} % Title is used for title page and header
\begin{document}
\maketitle % Create a title page
\makeheaders % Generate headers
\end{codeblock}

\subsection{Code and Keywords}

Use \verb|\code{text}| for inline code snippets and \verb|\keyword{text}| for keywords. For example:
Enter \code{cd directory} to change directories. The \keyword{while} loop continues until the condition is false.

\subsection{Navigation}

The \verb|\directions{a,b,c}| command creates a chain of directions with arrows:

\directions{Settings,Network,Advanced,Protocol}

\section{Special Environments}

\subsection{Code Blocks}

For multi-line code samples, use the \texttt{codeblock} environment:

\begin{codeblock}[language=python]
def calculate_checksum(data):
    total = 0
    for byte in data:
        total += byte
    return total & 0xFF
\end{codeblock}

This will prevent the code block from breaking across the page. If that's not the desired behavior, you can switch \texttt{codeblock} to the \texttt{lstlisting} environment as a drop-in replacement.

\subsection{Important Information}

Use the \texttt{important} environment to highlight critical information:

\begin{important}[Safety Warning]
Always wear safety goggles and gloves when handling chemicals in the laboratory.
\end{important}

\subsection{Extra Information}

The \texttt{extra} environment is useful for supplementary content:

\begin{extra}[Historical Context]
The TCP/IP protocol suite was developed in the 1970s by Vint Cerf and Bob Kahn.
\end{extra}

\subsection{Side-by-Side Content with Aside}

The \texttt{aside} environment allows you to place supplementary content to the right of your main document flow. Think of it as automatically placing an \texttt{extra} environment to the right of your current text without interrupting the natural flow of the document:

\begin{aside}[0.6,0.4]{Understanding IP Addresses}
\begin{leftcontent}
IP addresses are fundamental to network communication. Every device on a network needs a unique address to send and receive data. Modern networks primarily use IPv4 or IPv6 addresses.

When configuring a network interface, you'll need to set both the IP address and subnet mask. These work together to define the network segment your device belongs to.
\end{leftcontent}
\begin{rightcontent}
\begin{itemize}
\item IPv4: 32-bit address (e.g., 192.168.1.1)
\item IPv6: 128-bit address (e.g., 2001:...)
\item Special ranges reserved for private networks
\end{itemize}
\end{rightcontent}
\end{aside}

The main content continues to flow naturally here, uninterrupted by the supplementary information that appeared to the right.

\section{Questions and Assignments}

\subsection{Regular Questions}

Use the \texttt{question} environment for lab questions:

\begin{question}[Protocol Analysis]
What is the purpose of the TTL field in an IP packet header? How does it prevent infinite routing loops?
\end{question}

\begin{question}[Implementation]
Modify the provided checksum function to handle 16-bit words instead of bytes.
\end{question}

\subsection{Bonus Questions}

For extra credit or challenge questions, use the \texttt{bonusquestion} environment:

\begin{bonusquestion}[Performance Optimization]
How could you modify the checksum algorithm to utilize parallel processing?
\end{bonusquestion}

\section{Question Summary Section}

The \texttt{questionssection} environment serves 3 important purposes:
\begin{itemize}
\item It automatically adds a notice explaining that the questions are collected from throughout the lab manual for convenience and need only be answered once
\item It automatically resets both regular and bonus question counters after the section ends, ensuring proper numbering throughout the document.
\item It automatically uses Roman Numeral to separate it from the content sections
\end{itemize}

Example usage:

\begin{questionssection}
\begin{question}[Summary Question 1]
Explain the relationship between TCP sequence numbers and acknowledgment numbers.
\end{question}

\begin{question}[Summary Question 2]
What are the primary differences between UDP and TCP?
\end{question}
\end{questionssection}

\section{Utility Commands}

\subsection{TODOs}

Use \verb|\todo{text}| to mark sections that need attention:

\todo{Add diagram of three-way handshake}

\section{Style Guidelines}

\subsection{Tables}

For tables that span multiple pages, use the \texttt{longtable} environment. Always use booktabs-style rules (\verb|\toprule|, \verb|\midrule|, \verb|\bottomrule|) for professional formatting:

\begin{longtable}{lll}
\toprule
Protocol & Port & Description \\
\midrule
HTTP & 80 & Web traffic \\
HTTPS & 443 & Secure web traffic \\
SSH & 22 & Secure shell \\
\bottomrule
\end{longtable}

\subsection{Figures and Diagrams}

\begin{itemize}
\item Use TikZ for creating diagrams - the class includes libraries for:
  \begin{itemize}
  \item \code{calc} - For mathematical calculations
  \item \code{positioning} - For relative positioning of elements
  \item \code{arrows} - For customizable arrow tips
  \item \code{shadows.blur} - For adding depth to diagrams
  \end{itemize}
\item Always include captions and labels for figures:

\begin{figure}[htbp]
\centering
\begin{tikzpicture}
\node[draw] (a) {Client};
\node[draw, right=of a] (b) {Server};
\draw[->] (a) -- (b) node[midway, above] {Request};
\end{tikzpicture}
\caption{Basic Client-Server Communication}
\label{fig:client-server}
\end{figure}
\end{itemize}

\subsection{Cross-References}

\begin{itemize}
\item Use \verb|\autoref| instead of \verb|\ref| for automatic context in references
\item Always label significant sections, figures, and tables you might reference
\item Example: ``As shown in \verb|\autoref{fig:client-server}|...''
\end{itemize}

\subsection{Code Formatting}

\begin{itemize}
\item Use the provided \texttt{defaultstyle} for general code
\item Use \texttt{console} style for terminal output and commands
\item Code blocks automatically handle line breaks and maintain spacing
\end{itemize}

\subsection{Typography Guidelines}

\begin{itemize}
\item The class uses sans-serif fonts by default
\item Mathematical formulas should use \verb|amsmath| environments
\item Use \verb|\emph{}| for emphasis instead of underlining
\end{itemize}

\subsection{Page Layout}

\begin{itemize}
\item Create a title page using \verb|\maketitle| in combination with \verb|\clearpage|
\item Include a ToC using \verb|\tableofcontents| as a separate page with \verb|\clearpage|
\item Headers are automatically set up with the course number and title using the \verb|\makeheaders| which must be called in the document section after \verb|\title| has been defined
\item Figures and tables are allowed to occupy up to 90\% of the top of a page
\item The class automatically handles spacing around lists and environments
\item Use \verb|\vspace| sparingly - the class handles most spacing automatically
\item Use \verb|\clearpage| to create a new page when required for emphasis or formatting
\end{itemize}

\section{Tips for Best Results}

\begin{itemize}
\item Use the \texttt{aside} environment when you want to provide supplementary information without breaking document flow
\item Place the main content in \texttt{leftcontent} and supplementary material in \texttt{rightcontent}
\item Include a \texttt{questionssection} at the start of each lab to help students track their progress
\item Use the \texttt{important} environment sparingly to maintain its impact
\item Place code blocks inside questions when they relate directly to the assignment
\end{itemize}

\end{document}